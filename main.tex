% buildcmd: latexmk main.tex
\documentclass{article}
\usepackage[titletoc,title]{appendix}
\usepackage{amssymb}
\usepackage{graphicx}
\graphicspath{ {./images/} }
\title{Internationale Physik Olympiade 2022}
\author{Valentin Zwerschke}
\date{September, 2022}

\begin{document}
\maketitle
\section*{Aufgabe 2}
\subsection*{Teilaufgabe a)}
\subsubsection*{1. Epizentrum lokalisieren}
Die Wellen, primär und sekundär, breiten sich kreisförmig vom Epizentrum her aus. Andersherum betrachtet, kann das Signal an einem der Messstellen von jedem Punkt auf einem Kreis mit dem Radius $r$ liegen, 
der sich aus den Messzeiten der Ankunft Primär- und Sekundärwelle berechnen läßt. 
Das Epizentrum sollte im Schnittpunkt der jeweiligen Kreise um die Messstellen liegen. 
Zunächst berechne ich den Zusammenhang zwischen Radius und Ankuftszeitpunkten. 
Es gelten für die Radien der Primär- und Sekundärwellen die Formeln: 
\begin{center}
	$r_{p} = v_p \cdot t_1 = 5,5 km/s \cdot t_1$ \\ 
	$r_{s} = v_s \cdot t_2 = 3,3 km/s \cdot t_2$.
\end{center}
Hier bei ist $t_1$ die Laufzeit der Primärwelle und $t_2$ die Laufzeit der Sekundärwelle.\\
Die beiden Radien sollten gleich groß sein. Mit Hilfe des Gleichsetzungsverfahrens bekommt man die Gleichung: 
\space \space $v_p \cdot t_1  = v_s \cdot t_2$.\\
Anders geschrieben: $0 = (v_p-v_s) \cdot t_1  - v_s \cdot \Delta t$, 
wobei $\Delta t = t_2- t_1$ die Zeitdifferenz zwischen den zwei Signalen ist. 
Löst man nach $t_1$ auf, so bekommt man die Gleichung:
\begin{center}
	\item $t_1 = \frac{v_s }{v_p - v_s}\cdot \Delta t$ 
\end{center}
Aus dieser ergibt sich die gesuchte Beziehung zwischen Radius und $\Delta t$. 
\begin{center}
	\item $r = v_p \cdot t_1 = \frac{v_s}{v_p - v_s} \cdot v_p \cdot \Delta t$
\end{center}
Um das Epizentrum herauszubekommen, muss man mit Hilfe eines Zirkels jeweils einen Kreis mit dem entsprechenden Radius um die Messstelle zeichnen. 
Dann sieht man das Epizentrum an dem Schnittpunkt der vier Kreise (vgl. M1 A1). 
Wir benötigen die jeweiligen $\Delta t$ der Stationen. Dazu müssen wir die Differenz zwischen der P- und S-Wellensignalen berechnen.
\begin{itemize}
	\item MIYKNA: $\Delta t \approx 22,85s$ 
	\item ROKUGO: $\Delta t \approx 29,35s$ 
	\item N.KKWH: $\Delta t \approx 17,2s$ 
	\item N.KAKH: $\Delta t \approx 17,3s$ 
\end{itemize}
Als nächstes berechnen wir die jeweiligen Radien: 
\begin{itemize}
	\item $r_M = \frac{3,3km/s}{5,5 km/s - 3,3 km/s} \cdot 5,5 km/s \cdot 22,85 s = 188,5 km$
	\item $r_R = 242,1km$
	\item $r_{N.KK} = 141,9km$
	\item $r_{N.KA} = 142,7km$
\end{itemize}
Nun haben wir alle Radien und tragen die Kreise in die Karte um die Messstationen ein. 
Bei meiner Zeichnung (siehe Abb. M1.A1) schneiden sich die Kreise bei 38,9° Nord, 142,9° Ost. Dort sollte das Epizentrum liegen. 
\subsubsection*{2. Zeitpunkt berechnen}
Für die Berechnung des Zeitpunktes des Erdbebens verwende ich die Formel 
$t_0 = t_p - \frac{r}{v_p}$, 
wobei $t_p$ der Zeitpunkt des Eintreffens des Primärwellensignals ist. 
Ich berechne den Zeitpunkt aus den Werten $t_0$ aller vier Stationen und berechne den Mittelwert. 
In der folgenden Tabelle habe ich die Ergebnisse zusammengetragen.
\begin{center}
\begin{tabular}{l|l|l|l|l|l|l|l}
		   & $t_{p}$         & $t_s$         & $\Delta t$ & $r$      & $\frac{r}{v_p}$    & $\frac{r}{v_s}$    & $t_0$          \\\hline
	MIYKNA & 14:46:46,71 & 14:47:9,56  & 22,85   & 188,51 & 34,28 & 57,13 & 14:46:12,44 \\
	ROKUGO & 14:46:54,18 & 14:47:23,53 & 29,35   & 242,14 & 44,03 & 73,38 & 14:46:10,16 \\
	N.KKWH & 14:46:40,13 & 14:46:57,33 & 17,2    & 141,90 & 25,80 & 43,00 & 14:46:14,33 \\
	N.KAKH & 14:46:40,57 & 14:46:57,87 & 17,3    & 142,73 & 25,95 & 43,25 & 14:46:14,62 
\end{tabular}
\end{center}
Rechnet man jetzt den Mittelwert von den 4 $t_0$ Werten aus, kann man davon ausgehen, 
dass das Erdbeben um \textbf{14:46:12,89} stattgefunden hat.


\subsection*{Teilaufgabe b)}
Um das Verhältnis der Energien herauszufinden, drücke ich alle Energien über das gegebene Energieverhältnis durch $E_6$, der Energie eines Bebens der Stärlke 6, aus. 
So kann man im Verhältnis der gesuchten Energien $E_6$ kürzen und bekommt die gesuchte Verhältniszahl.\\
Die Energie aller auftretenden Beben der Stärke 6-8 beschreibt die folgende Summe über die jeweiligen Energien multipliziert mit der Auftretenswahrscheinlichkeit pro Jahr. 
\begin{center}
$E_{alle} = 1 \cdot E_8 + 9 \cdot E_7 + 90 \cdot E_6$. 
\end{center}
Nachdem wir nun $E_{alle}$ haben, muss ich es nur noch mit $E_9$ (auch durch $E_6$ ausgedrückt) ins Verhältnis setzen, d.h.:
\begin{center}
$\frac{E_9}{E_{alle}} = \frac{E_6 \cdot 10^{\frac{3}{2} (9 - 6)}}{E_6 \cdot 10^{\frac{3}{2}(8-6)} + 9 \cdot E_6 \cdot 10^{\frac{3}{2}(7-6)} + 90 \cdot E_6}
= \frac{10^\frac{9}{2}}{10^3 + 0 \cdot 10^\frac{3}{2} + 90} \approx 23$
\end{center}
Das Beben war hatte somit eine ca. 23-fach hähere Energie. 
\section*{Aufgabe 3}
\subsection*{Teilaufgabe a) Schwerpunktbahn}
Die Bewegung des beiden Punkte kann man sich als Überlagerungen der Schwerpunktbewegung mit jeweils einer Kreisbewegung vorstellen, wobei die Schwerpunktbewegung in x-Richtung eine gleichförmige ist mit 
Anfangsposition 0 und in y-Richtung eine Überlagerung aus freiem Fall und gleichförmiger Bewegung mit Anfangsposition $y_0$.
\begin{center}
	$x_s (t)= v_{x0}\cdot t$ \\
	$y_s (t) = v_{y0}\cdot t + y_0 -0.5 g t^2$
\end{center}   
Die Überlagerten Kreisbewebungen sind
\begin{center}
	$x_p (t) = p \cdot (-\sin(\omega t))$ ; $x_q (t)= q \cdot (\sin(\omega t)) $\\
	$y_p (t) = p \cdot (\cos(\omega t))$ ; $y_q (t)= q \cdot (-\cos(\omega t)) $
\end{center} 
wobei $\omega = 2\pi f $ die Kreisfrequenz der Kreisbewegung mit der gesuchten Frequenz $f$ ist.\\
Zunächst rechne ich die Bewegung durch den freien Fall heraus.  
\begin{center}
	\includegraphics[scale=0.6]{Kurve-P-Punkt-ohne-freien-Fall.png}
	\includegraphics[scale=0.6]{Kurve-Q-Punkt-ohne-freien-Fall.png}
\end{center}





\section*{Aufgabe 4}
\subsection*{Teilaufgabe a) Experiment - Messwerte}
Ich habe unsere Teekanne (ca. 1 Liter Fassungsvermögen) mit heißem Wasser aus dem Wasserkocher befüllt und anstelle des Deckekls ein Thermometer aufgesetzt, welches zufälligerweise ideal passte. 
Das verwendete Heizungsventilthermometer hat einen Metallstab als Fühler, der ungefähr in der Mitte der Tekanne war. 
Zunächst alle 5 Minuten, später ungefähr alle 10 Minuten habe ich die Temperatur abgelesen. Die Ergebnisse stehen in den ersten zwei Spalten der folgenden Tabelle.\\ 
\newpage
\begin{tabular}{cccc}
	Zeit t {[}s{]} & Temp $\vartheta$ {[}°C{]}              & \begin{tabular}[c]{@{}l@{}}$\frac{\vartheta-\vartheta_u}{\vartheta_o-\vartheta_u}$\end{tabular} & $-\ln{\frac{\vartheta-\vartheta_U}{\vartheta_O-\vartheta_U}}$ \\\hline
	0              & 56,0 						  & 1,00                                                                 & 0,00     \\
	5              & 54,3                         & 0,95                                                                 & 0,05     \\
	10             & 53,0                         & 0,92                                                                 & 0,08     \\
	15             & 52,0                         & 0,89                                                                 & 0,11     \\
	20             & 50,0                         & 0,84                                                                 & 0,18     \\
	25             & 48,0                         & 0,78                                                                 & 0,24     \\
	30             & 47,0                         & 0,76                                                                 & 0,28     \\
	35             & 45,0                         & 0,70                                                                 & 0,35     \\
	40             & 44,0                         & 0,68                                                                 & 0,39     \\
	45             & 43,0                         & 0,65                                                                 & 0,43     \\
	50             & 42,3                         & 0,63                                                                 & 0,46     \\
	55             & 41,0                         & 0,59                                                                 & 0,52     \\
	60             & 40,0                         & 0,57                                                                 & 0,57     \\
	65             & 39,5                         & 0,55                                                                 & 0,59     \\
	70             & 39,0                         & 0,54                                                                 & 0,62     \\
	75             & 38,0                         & 0,51                                                                 & 0,67     \\
	80             & 37,5                         & 0,50                                                                 & 0,69     \\
	85             & 36,0                         & 0,46                                                                 & 0,78     \\
	90             & 36,0                         & 0,46                                                                 & 0,78     \\
	95             & 35,0                         & 0,43                                                                 & 0,84     \\
	100            & 34,0                         & 0,41                                                                 & 0,90     \\
	105            & 33,5                         & 0,39                                                                 & 0,94     \\
	110            & 33,0                         & 0,38                                                                 & 0,97     \\
	115            & 32,3                         & 0,36                                                                 & 1,02     \\
	120            & 32,0                         & 0,35                                                                 & 1,05     \\
	125            & 31,5                         & 0,34                                                                 & 1,09     \\
	130            & 31,0                         & 0,32                                                                 & 1,13     \\
	135            & 30,5                         & 0,31                                                                 & 1,17     \\
	140            & 30,0                         & 0,30                                                                 & 1,21     \\
	145            & 29,9                         & 0,29                                                                 & 1,22     \\
	150            & 29,7                         & 0,29                                                                 & 1,24     \\
	155            & 29,5                         & 0,28                                                                 & 1,26     \\
	160            & 28,7                         & 0,26                                                                 & 1,34     \\
	165            & 28,3                         & 0,25                                                                 & 1,38     \\
	170            & 28,0                         & 0,24                                                                 & 1,41     \\
	175            & 27,5                         & 0,23                                                                 & 1,47     \\
	180            & 27,3                         & 0,22                                                                 & 1,49     \\
	185            & 27,2                         & 0,22                                                                 & 1,51     \\
	190            & 26,5                         & 0,20                                                                 & 1,60 
\end{tabular}
\begin{tabular}{cccc}
	Zeit t {[}s{]} & Temp $\vartheta$ {[}°C{]}              & \begin{tabular}[c]{@{}l@{}}$\frac{\vartheta-\vartheta_u}{\vartheta_o-\vartheta_u}$\end{tabular} & $-\ln{\frac{\vartheta-\vartheta_U}{\vartheta_O-\vartheta_U}}$ \\\hline
	200            & 25,9                         & 0,19                                                                 & 1,68     \\
	218            & 24,5                         & 0,15                                                                 & 1,91     \\
	228            & 24,1                         & 0,14                                                                 & 1,98     \\
	238            & 24,0                         & 0,14                                                                 & 2,00     \\
	248            & 23,7                         & 0,13                                                                 & 2,06     \\
	265            & 22,3                         & 0,09                                                                 & 2,42     \\
	275            & 22,1                         & 0,08                                                                 & 2,48     \\
	286            & 22,0                         & 0,08                                                                 & 2,51     \\
	297            & 21,9                         & 0,08                                                                 & 2,55     \\
	308            & 21,5                         & 0,07                                                                 & 2,69     \\
	319            & 20,5                         & 0,04                                                                 & 3,21     \\
	330            & 20,3                         & 0,04                                                                 & 3,35     \\
	341            & 20,1                         & 0,03                                                                 & 3,52     \\
	352            & 20,0                         & 0,03                                                                 & 3,61     \\
	363            & 19,9                         & 0,02                                                                 & 3,72     \\
	374            & 19,8                         & 0,02                                                                 & 3,83     \\
	385            & 19,8                         & 0,02                                                                 & 3,83     
\end{tabular}
\newpage
Die Abkühlkurve meines Experiments sieht wie folgt aus. 
\begin{center}
	\includegraphics[scale=0.6]{Graph1.png}
\end{center}


\subsection*{Teilaufgabe b) - Bestimmung Zeitkonstante}
Zunächst will ich zeigen, dass die gewonnene Abkühlkurve gemäß der Newtonschon Formel 
\begin{center}
	$\vartheta (t) = \vartheta_u + (\vartheta_o - \vartheta_u) \cdot e^{-t/\tau}$ 
\end{center}
verläuft. Dazu forme ich die Gleichung wie folgt um
\begin{center}
	$\frac{\vartheta (t) - \vartheta_u}{\vartheta_o - \vartheta_u} = e^{-t/\tau}$ 
\end{center}
Ich nehme den Logarithmus auf beiden Seiten der Gleichung und erhalte eine Geradengleichung, 
aus deren Steigung sich die Abkühlkonstante $\tau$ berechnen lässt.
\begin{center}
	$-\ln{\frac{\vartheta (t) - \vartheta_u}{\vartheta_o - \vartheta_u}} = \frac{1}{\tau}\cdot t$ 
\end{center}
Die Ergebnisse des linken Terms habe ich ebenfalls in der obigen Tabelle eingetragen. 
Wie man in der folgenden Graphik sieht, liegen die Messwerte tatsächlich auf einer Geraden. Somit habe ich gezeigt, dass der Abkühlprozess tatsächlich dem Newtonschon Gesetz folgt.  
\begin{center}
	\includegraphics[scale=0.6]{Graph2.png}
\end{center}
Die Steigung der Geraden lese ich ab zu $\frac{2,2}{250}\cdot\frac{1}{s}=\frac{0,0088}{s}$.
Der Kehrwert ist die gesuchte Abkühlkonstante $\tau = 113,6 s$.
\subsection*{Teilaufgabe c) - Zubereitung Grüner Tee}
Ich verwende das Newtonsche Abkühlungsgesetz, diesmal mit $\vartheta_o=100^\circ C$.
\begin{center}
	$70^\circ C = \vartheta (t) = \vartheta_u + (100^\circ C - \vartheta_u ) \cdot e^{-t/\tau}$ 
\end{center}
Es ergibt sich mit dem oben bestimmten Wert für $\tau$ und $\vartheta (t) = 70^\circ C$
\begin{center}
	$t=-\tau \ln{\frac{\vartheta (t) - \vartheta_u}{ 100^\circ C - \vartheta_u}} \approx 53 s$
\end{center}
Dieses Ergebnis ist wie in der Aufgabe beschrieben eher ungenau, da das Newtonsche Abkühlgesetz nur für kleine Temperaturdifferenzen gilt und 
die Größe $\tau$ sicherlich auch eine gewisse Abhängigkeit von der Temperatur hat. In der Literatur findet man z.B. eine inverse Abhängigkeit von der Wärmekapazität des Wassers.  
\\
Da das Wasser knapp unter dem Siedepunkt stärker verdampft als bei kleinen Temperaturen entzieht die Verdampfungswärme dem Teewasser bei den hohen Temperaturen zusätzliche Energie, 
was ein schnelleres Erkalten des Teewasseres erwarten lässt. 
Die Newtonsche Abkühlungskurve beschreibt ja lediglich den Wärmeübergang durch Wärmeleitung ins Gefäß und die umgebende Luft und durch die Abstrahlung. 
Energieänderung durch Verdampfen ist nicht enthalten.      
\end{document}
