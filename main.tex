% buildcmd: latexmk main.tex
\documentclass{article}
\usepackage[titletoc,title]{appendix}
\usepackage{amssymb}
\title{Internationale Physik Olympiade 2022}
\author{Valentin Zwerschke}
\date{September, 2022}

\begin{docuent}
\maketitle
\section*{Aufgabe 2}
\subsection*{Teilaufgabe a)}
\subsubsection*{1. Epizentrum lokalisieren}
Es gelten fuer die Radien der primaer und sekundaer Geschwindigkeiten die Formeln: 
$s_{primaer} = 5,5 km/s \cdot t_1 ; s_{sekundaer} = 3,3 km/s \cdot t_2$.\\
Stellt man diese Formeln mithilfe des Gleichsetzungsverfahrens um bekommt man dann die Formel: \space \space $0 = 5,5 km/s \cdot t_1 - 3,3 km\h \cdot t_2$.\\Vereinfacht geschrieben: $0 = 2,2 km/s \cdot t_1  + 3,3 km/s \cdot \Delta t$
Loest man hier also nach $t_1$ auf bekommt man die Gleichung: $t_1 = - \frac{3,3}{2,2} \cdot \Delta t$.\\
Die zwei wichtigen Formeln die wir hieraus bekommen sind: 
\begin{itemize}
	\item $t_1 = \frac{v_s \cdot \Delta t}{v_p - v_s}$
	\item $r = v_p \cdot t_1 = \frac{v_s}{v_p - v_s} \cdot v_p \cdot \Delta t$
\end{itemize}
Um also das Epizentrum herauszubekommen muss man mithilfe eines Zirkels jeweils einen Kreis mit dem Radius zeichnen und dann sieht man das Epizentrum an dem Schnittpunkt der vier Kreise.\\
Dazu benoetigen wir zunaechst die jeweiligen $\Delta t$ der Stationen. Dazu muessen wir nur die Differenz zwischen der P- und S-Wellen berechnen.
\begin{itemize}
	\item MIYKNA: $\Delta t \approx 22,85s$ 
	\item ROKUGO: $\Delta t \approx 29,35s$ 
	\item N.KKWH: $\Delta t \approx 17,2s$ 
	\item N.KAKH: $\Delta t \approx 17,3s$ 
\end{itemize}
Als naechstes die jeweiligen Radien ausrechnen: 
\begin{itemize}
	\item $r_M = \frac{3,3km/s}{5,5 km/s - 3,3 km/s} \cdot 5,5 km/s \cdot 22,85 s = 188,51 km$
	\item $r_R = 242,1375km$
	\item $r_{N.KK} = 141,9km$
	\item $r_{N.KA} = 142,725km$
\end{itemize}
Nun haben wir alle Radien und muessen sie in die Karte einzeichnen. Bei meiner Zeichnung (siehe M1.A1) schneiden sich die Kreise bei 38,9° Nord, 142,9° Ost.
\newpage
\subsubsection*{2. Zeitpunkt errechnen}
Um den Zeitpunkt des Erdbebens herauszufinden, muss man einfach den Mittelwert der Formel fuer $t_0$ an den jeweiligen Stationen nehmen. Diese lautet wiefolgt: $t_0 = t_p - \frac{r}{v_p}$\\Mit $t_p$ ist der Ankunftszeitpunkt der Primaerwelle an der jeweiligen Station gemeint.\\\\
\begin{tabular}{l|l|l|l|l|l|l|l}
		   & $t_{p}$         & $t_s$         & $\Delta t$ & $r$      & $\frac{r}{v_p}$    & $\frac{r}{v_s}$    & $t_0$          \\\hline
	MIYKNA & 14:46:46,71 & 14:47:9,56  & 22,85   & 188,51 & 34,28 & 57,13 & 14:46:12,44 \\
	ROKUGO & 14:46:54,18 & 14:47:23,53 & 29,35   & 242,14 & 44,03 & 73,38 & 14:46:10,16 \\
	N.KKWH & 14:46:40,13 & 14:46:57,33 & 17,2    & 141,90 & 25,80 & 43,00 & 14:46:14,33 \\
	N.KAKH & 14:46:40,57 & 14:46:57,87 & 17,3    & 142,73 & 25,95 & 43,25 & 14:46:14,62 
\end{tabular}
\\\\Rechnet man jetzt den Mittelwert von den 4 $t_0$ Werten aus, kann man davon ausgehen, dass das Erdbeben um \textbf{14:46:12,89} stattfindet.
\subsection*{Teilaufgabe b)}
Um das Formel Verhaeltnis herauszufinden sollte man alles mit dem $E_6$ Verhaeltnis ausdruecken um es somit zu kuerzen und das Ergebnis herauszufinden.\\
$E_{alle} = 1 \cdot E_8 + 9 \cdot E_7 + 90 \cdot E_6$. Nachdem wir nun $E_alle$ haben muss es nur noch $E_9$ teilen, dh.:
$\frac{E_9}{E_{alle}} = \frac{E_6 \cdot 10^{\frac{3}{2} (9 - 6)}}{E_6 \cdot 10^{\frac{3}{2}(8-6)} + 9 \cdot E_6 \cdot 10^{\frac{3}{2}(7-6)} + 90 \cdot E_6}$.\\Jetzt kann man $E_6$ rauskuerzen.
$\frac{E_9}{E_{alle}} = \frac{10^\frac{9}{2}}{10^3 + 0 \cdot 10^\frac{3}{2} + 90} \approx 23$


\end{document}
