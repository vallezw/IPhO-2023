% buildcmd: latexmk main.tex
\documentclass{article}
\usepackage[titletoc,title]{appendix}
\usepackage{amssymb}
\title{Internationale Physik Olympiade 2022}
\author{Valentin Zwerschke}
\date{September, 2022}

\begin{document}
\maketitle
\section*{Aufgabe 2}
\subsection*{Teilaufgabe a)}
\subsubsection*{1. Epizentrum lokalisieren}
Die Wellen, primär und sekundär, breiten sich kreisförmig vom Epizentrum aus aus. Andersherum betrachtet, kann das Signal an einem der Messstellen von jedem Punkt auf einem Kreis mit dem Radius $r$ liegen, 
der sich aus den Messzeiten der Ankunft Primär- und Sekundärwelle berechnen läßt. 
Das Epizentrum sollte im Schnittpunkt der jeweiligen Kreise um die Messstellen liegen. 
Zunächst berechne ich den Zusammenhang zwischen Radius und Ankuftszeitpunkten. 
Es gelten für die Radien der Primär- und Sekundärwellen die Formeln: 
\begin{center}
$r_{primär} = v_p \cdot t_1= 5,5 km/s \cdot t_1 \\ 
r_{sekundär} = v_s\cdot t_2 = 3,3 km/s \cdot t_2$.
\end{center}
Hier bei ist $t_1$ die Laufzeit der Primärwelle und $t_2$ die Laufzeit der Sekundärwelle.\\
Die beiden Radien sollten gleich groß sein. Mit Hilfe des Gleichsetzungsverfahrens bekommt man die Gleichung: 
\space \space $v_p \cdot t_1  = v_s \cdot t_2$.\\
Anders geschrieben: $0 = (v_p-v_s) \cdot t_1  - v_s \cdot \Delta t$, 
wobei $\Delta t = t_2- t_1$ die Zeitdifferenz zwischen den zwei Signalen ist. 
Löst man nach $t_1$ auf, so bekommt man die Gleichung:
\begin{center}
	\item $t_1 = \frac{v_s }{v_p - v_s}\cdot \Delta t$
\end{center}
Aus dieser ergibt sich die gesuchte Beziehung zwischen Radius und $\Delta t$. 
\begin{center}
	\item $r = v_p \cdot t_1 = \frac{v_s}{v_p - v_s} \cdot v_p \cdot \Delta t$
\end{center}
Um das Epizentrum herauszubekommen, muss man mit Hilfe eines Zirkels jeweils einen Kreis mit dem entsprechenden Radius um die Messstelle zeichnen. 
Dann sieht man das Epizentrum an dem Schnittpunkt der vier Kreise.\\
Wir benötigen die jeweiligen $\Delta t$ der Stationen. Dazu müssen wir die Differenz zwischen der P- und S-Wellensignalen berechnen.
\begin{itemize}
	\item MIYKNA: $\Delta t \approx 22,85s$ 
	\item ROKUGO: $\Delta t \approx 29,35s$ 
	\item N.KKWH: $\Delta t \approx 17,2s$ 
	\item N.KAKH: $\Delta t \approx 17,3s$ 
\end{itemize}
\newpage
Als nächstes die jeweiligen Radien ausrechnen: 
\begin{itemize}
	\item $r_M = \frac{3,3km/s}{5,5 km/s - 3,3 km/s} \cdot 5,5 km/s \cdot 22,85 s = 188,51 km$
	\item $r_R = 242,1375km$
	\item $r_{N.KK} = 141,9km$
	\item $r_{N.KA} = 142,725km$
\end{itemize}
Nun haben wir alle Radien und muessen sie in die Karte einzeichnen. Bei meiner Zeichnung (siehe M1.A1) schneiden sich die Kreise bei 38,9° Nord, 142,9° Ost.
\subsubsection*{2. Zeitpunkt berechnen}
Um den Zeitpunkt des Erdbebens herauszufinden, muss man einfach den Mittelwert der Formel fuer $t_0$ an den jeweiligen Stationen nehmen. Diese lautet wiefolgt: $t_0 = t_p - \frac{r}{v_p}$\\Mit $t_p$ ist der Ankunftszeitpunkt der Primaerwelle an der jeweiligen Station gemeint.\\\\
\begin{tabular}{l|l|l|l|l|l|l|l}
		   & $t_{p}$         & $t_s$         & $\Delta t$ & $r$      & $\frac{r}{v_p}$    & $\frac{r}{v_s}$    & $t_0$          \\\hline
	MIYKNA & 14:46:46,71 & 14:47:9,56  & 22,85   & 188,51 & 34,28 & 57,13 & 14:46:12,44 \\
	ROKUGO & 14:46:54,18 & 14:47:23,53 & 29,35   & 242,14 & 44,03 & 73,38 & 14:46:10,16 \\
	N.KKWH & 14:46:40,13 & 14:46:57,33 & 17,2    & 141,90 & 25,80 & 43,00 & 14:46:14,33 \\
	N.KAKH & 14:46:40,57 & 14:46:57,87 & 17,3    & 142,73 & 25,95 & 43,25 & 14:46:14,62 
\end{tabular}
\\\\Rechnet man jetzt den Mittelwert von den 4 $t_0$ Werten aus, kann man davon ausgehen, dass das Erdbeben um \textbf{14:46:12,89} stattfindet.
\subsection*{Teilaufgabe b)}
Um das Formel Verhaeltnis herauszufinden sollte man alles mit dem $E_6$ Verhaeltnis ausdruecken um es somit zu kuerzen und das Ergebnis herauszufinden.\\
$E_{alle} = 1 \cdot E_8 + 9 \cdot E_7 + 90 \cdot E_6$. Nachdem wir nun $E_alle$ haben muss es nur noch $E_9$ teilen, dh.:
$\frac{E_9}{E_{alle}} = \frac{E_6 \cdot 10^{\frac{3}{2} (9 - 6)}}{E_6 \cdot 10^{\frac{3}{2}(8-6)} + 9 \cdot E_6 \cdot 10^{\frac{3}{2}(7-6)} + 90 \cdot E_6}$.\\Jetzt kann man $E_6$ rauskuerzen.
$\frac{E_9}{E_{alle}} = \frac{10^\frac{9}{2}}{10^3 + 0 \cdot 10^\frac{3}{2} + 90} \approx 23$

\section*{Aufgabe 4}
\subsection*{Teilaufgabe a) - Messwerte}
\begin{tabular}{llll}
	Zeit t {[}s{]} & Temp T {[}°C{]}              & \begin{tabular}[c]{@{}l@{}}(T-T\_U)/\\      (T\_0-T\_U)\end{tabular} & -ln(...) \\\hline
	0              & 56,0 						  & 1,00                                                                 & 0,00     \\
	5              & 54,3                         & 0,95                                                                 & 0,05     \\
	10             & 53,0                         & 0,92                                                                 & 0,08     \\
	15             & 52,0                         & 0,89                                                                 & 0,11     \\
	20             & 50,0                         & 0,84                                                                 & 0,18     \\
	25             & 48,0                         & 0,78                                                                 & 0,24     \\
	30             & 47,0                         & 0,76                                                                 & 0,28     \\
	35             & 45,0                         & 0,70                                                                 & 0,35     \\
	40             & 44,0                         & 0,68                                                                 & 0,39     \\
	45             & 43,0                         & 0,65                                                                 & 0,43     \\
	50             & 42,3                         & 0,63                                                                 & 0,46     \\
	55             & 41,0                         & 0,59                                                                 & 0,52     \\
	60             & 40,0                         & 0,57                                                                 & 0,57     \\
	65             & 39,5                         & 0,55                                                                 & 0,59     \\
	70             & 39,0                         & 0,54                                                                 & 0,62     \\
	75             & 38,0                         & 0,51                                                                 & 0,67     \\
	80             & 37,5                         & 0,50                                                                 & 0,69     \\
	85             & 36,0                         & 0,46                                                                 & 0,78     \\
	90             & 36,0                         & 0,46                                                                 & 0,78     \\
	95             & 35,0                         & 0,43                                                                 & 0,84     \\
	100            & 34,0                         & 0,41                                                                 & 0,90     \\
	105            & 33,5                         & 0,39                                                                 & 0,94     \\
	110            & 33,0                         & 0,38                                                                 & 0,97     \\
	115            & 32,3                         & 0,36                                                                 & 1,02     \\
	120            & 32,0                         & 0,35                                                                 & 1,05     \\
	125            & 31,5                         & 0,34                                                                 & 1,09     \\
	130            & 31,0                         & 0,32                                                                 & 1,13     \\
	135            & 30,5                         & 0,31                                                                 & 1,17     \\
	140            & 30,0                         & 0,30                                                                 & 1,21     \\
	145            & 29,9                         & 0,29                                                                 & 1,22     \\
	150            & 29,7                         & 0,29                                                                 & 1,24     \\
	155            & 29,5                         & 0,28                                                                 & 1,26     \\
	160            & 28,7                         & 0,26                                                                 & 1,34     \\
	165            & 28,3                         & 0,25                                                                 & 1,38     \\
	170            & 28,0                         & 0,24                                                                 & 1,41     \\
	175            & 27,5                         & 0,23                                                                 & 1,47     \\
	180            & 27,3                         & 0,22                                                                 & 1,49     \\
	185            & 27,2                         & 0,22                                                                 & 1,51     \\
	190            & 26,5                         & 0,20                                                                 & 1,60     \\
	200            & 25,9                         & 0,19                                                                 & 1,68     \\
	218            & 24,5                         & 0,15                                                                 & 1,91     \\
	228            & 24,1                         & 0,14                                                                 & 1,98     \\
	238            & 24,0                         & 0,14                                                                 & 2,00     \\
	248            & 23,7                         & 0,13                                                                 & 2,06     \\
	265            & 22,3                         & 0,09                                                                 & 2,42     \\
	275            & 22,1                         & 0,08                                                                 & 2,48     \\
	286            & 22,0                         & 0,08                                                                 & 2,51     \\
	297            & 21,9                         & 0,08                                                                 & 2,55     \\
	308            & 21,5                         & 0,07                                                                 & 2,69     \\
	319            & 20,5                         & 0,04                                                                 & 3,21     \\
	330            & 20,3                         & 0,04                                                                 & 3,35     \\
	341            & 20,1                         & 0,03                                                                 & 3,52     \\
	352            & 20,0                         & 0,03                                                                 & 3,61     \\
	363            & 19,9                         & 0,02                                                                 & 3,72     \\
	374            & 19,8                         & 0,02                                                                 & 3,83     \\
	385            & 19,8                         & 0,02                                                                 & 3,83     \\
\end{tabular}

\subsection*{Teilaufgabe b)}
Zunaechst wollen wir die Formel $\vartheta (t) = \vartheta_u + (\vartheta_0 - \vartheta_U) \cdot e^{-t/r}$
\end{document}
